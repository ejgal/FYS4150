\section{Discussion}

The values for the confidence intervals around the T$_c$(L=$\infty$) assumes
that the values for T$_c$(L) are certain. This is not correct, since we
introduce uncertainties when running the simulations and by the method we use to
estimate T$_c$(L). As shown in \cref{fig:error_L2} our model exhibited relative
errors on the order of \num{1e-3} compared to the analytic solution using a 2x2
grid. If the errors on a larger grid are comparable this should not be a
significant source of error.

We tested two other methods for estimating T$_C(L)$. Selecting the n largest
points, and taking a rolling mean with a window of n for varying n. This gave
similar results for n reasonable small compared to the 300 different temperature
we ran experiments for. This gave similar results for T$_C(L)$ and
T$_C(L=\infty)$, so this does not seem to be a significant source of error
either.

Compared to Onsager's results (\cref{eq:analytical}) we got an absolute error
of \num{4.54e-04} and a relative error of \num{2.00e-4}.
