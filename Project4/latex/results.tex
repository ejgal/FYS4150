\section{Results}




\subsection{Probability distribution}

The probability distribution of energy (\cref{fig:distribution}) shows that
at low temperature almost all of the states are in a low energy state.
As the temperature and the variance increases the distribution becomes more
spread out.

\begin{figure}[H]
  \centering
  \includegraphics[width=\textwidth]{../figures/distribution.png}
  \caption{Probability distribution of energy, scaled with number of spins. L=20}
  \label{fig:distribution}
\end{figure}




\subsection{Phase transitions}

T $\in \brak{2.0,2.8}$, dT = \num{4e-3} and L $\in \brak{40,60,80,100}$,
\num{1e6} MC cycles,
showed (\cref{fig:phase_E,fig:phase_Mabs,fig:phase_cv,fig:phase_suscept})
indications of a phase transition for T $\in \brak{2.2,2.35}$. The simulations
with higher grid size exhibits sharper changes, meaning that they capture
the phase transition better.

To find the critical temperature we did a run in the interesting interval with a
smaller dT (\num{5e-4}). We fitted fourth order polynomials to $C_v$ and found
the critical temperature for each grid size (Tc$_{L}$) (See
critical\_temperature.py) \footnote{We also tried to find the T$c$(L) by two
other methods. 1. Selecting the 3-5 largest points. 2. Taking a rolling mean
with a window of 3-5. This gave similar results for T$_C(L=\infty)$, but was
more cumbersome to implement.}. Using the finite size scaling relation
\cref{eq:scaling} with $\nu$ = 1, we found the parameter a by doing a linear
regression with the four different T$_c$(L).

 \begin{equation}
   \label{eq:scaling}
   T_C(L=\infty) = T_C(L) - aL^{-1/\nu}
 \end{equation}

 This gave us the results (with a 99\% confidence) of $Tc=2.26709 \pm 0.00237$.
 Comparing with the analytical solution gave us the relative error
 \cref{eq:analytical} $\epsilon_r = 0.00092$ with the 99\% confidence interval
 $\brak{0.00012,0.00197}$.

\begin{equation}
  \label{eq:analytical}
  T_C(L=\infty) = \frac{2}{\ln(1 + \sqrt2)} \approx 2.269
\end{equation}


\begin{figure}[H]
  \centering
  \includegraphics[width=\textwidth]{../figures/phase_E.png}
  \caption{Mean energy. Some small differences between grid sizes. Run with T $\in \brak{2.0,2.8}$, dT = \num{4e-3} and
  \num{1e6} MC cycles}
  \label{fig:phase_E}
\end{figure}

\begin{figure}[H]
  \centering
  \includegraphics[width=\textwidth]{../figures/phase_Mabs.png}
  \caption{Mean magnetisation.  Run with T $\in \brak{2.0,2.8}$, dT = \num{4e-3} and
  \num{1e6} MC cycles}
  \label{fig:phase_Mabs}
\end{figure}



\begin{figure}[H]
  \centering
  \includegraphics[width=\textwidth]{../figures/phase_cv.png}
  \caption{Heat capacity. Run with T $\in \brak{2.0,2.8}$, dT = \num{4e-3} and
  \num{1e6} MC cycles}
  \label{fig:phase_cv}
\end{figure}


\begin{figure}[H]
  \centering
  \includegraphics[width=\textwidth]{../figures/phase_suscept.png}
  \caption{Susceptibility. Run with T $\in \brak{2.0,2.8}$, dT = \num{4e-3} and
  \num{1e6} MC cycles}
  \label{fig:phase_suscept}
\end{figure}
