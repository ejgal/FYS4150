\section{Results}

\subsection{Phase transitions}

We did a production run to look for signs of phase transitions varying T between
2.0 and 2.8 with a .
The results (\cref{fig:phase}) indicated a phase transition for T $\in
\brak{2.2,2.35}$. The simulations with higher grid size exhibits sharper
changes, meaning that they will capture the phase transition better.

\begin{figure}[H]
  \centering
  \includegraphics[width=\textwidth]{../figures/phase.pdf}
  \caption{Simulating for various temperatures looking for phase transitions.\\
  T $\in \brak{2.0,2.8}$, dT = \num{4e-3}, \num{1e6} MC cycles, delay=50000}
  \label{fig:phase}
\end{figure}


To better capture the peaks of $C_v$ and $\chi$ we did another
run with dT = \num{5e-4} for T $\in \brak{2.2, 2.35}$.
 We fitted sixth order polynomials to $C_v$ and found
the critical temperature Tc$_{L}$ for each grid size (\ref{fig:polyfit}).


Using the finite size scaling relation in
\cref{eq:scaling} with $\nu$ = 1, we estimated $T_c(L=\infty)$
by doing a linear
regression with the four different T$_C$(L), using $1/L$ as the independent
variable, and extracting the intercept (\cref{fig:lin_reg}).
\footnote{Implemented in critical\_temperature.py}. The results are shown in
\cref{tab:results}.

 \begin{equation}
   \label{eq:scaling}
   T_C(L=\infty) = T_C(L) - aL^{-1/\nu}
 \end{equation}


 \begin{figure}[ht]
   \begin{subfigure}[t]{.5\textwidth} % top align
     \centering
     % include first image
     \includegraphics[width=\linewidth]{../figures/fit.pdf}
     \caption{Fit of sixth order polynomials (lines) to the measurements
     of heat capacity (circles).}
     \label{fig:polyfit}
   \end{subfigure}
   \hfill
   \begin{subfigure}[t]{.5\textwidth}
     \centering
     \includegraphics[width=\linewidth]{../figures/Tc_fit.pdf}
     \caption{Linear regression on the critical temperatures for L $\in \brak{[40,60,80,100]}$
     to extrapolate T$_C(L=\infty)$.}
     \label{fig:lin_reg}
   \end{subfigure}
   \label{fig:test}
 \end{figure}


 % This gave us the result of $Tc=2.26709 \pm 0.00237$.
 % Comparing with the analytical solution gave us the relative error
 % \cref{eq:analytical} $\epsilon_r = 0.00092$ with the 99\% confidence interval
 % $\brak{0.00012,0.00197}$.

 \begin{table}[htp]
 \centering
 \begin{tabular}{|l|l|l|ll}
 \cline{1-3}
                &        & 99\% confidence interval  &  &  \\ \cline{1-3}
 T$_C(L=\infty)$          & 2.26709   & {[}2.26472,2.26946{]}     &  &  \\ \cline{1-3}
 Relative error & 9.243e-04 & {[}1.210e-04,1.970e-03{]} &  &  \\ \cline{1-3}
 \end{tabular}
 \caption{Estimations of the critical temperature. Relative errors obtained by
 comparison with Lars Onsager's analytical solution \cref{eq:analytical}.}
 \label{tab:results}
 \end{table}


\begin{equation}
  \label{eq:analytical}
  T_C(L=\infty) = \frac{2}{\ln(1 + \sqrt2)} \approx 2.269
\end{equation}
