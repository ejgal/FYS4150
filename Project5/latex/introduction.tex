\section{Introduction}
The concept of Rossby waves was first introduced by the Swedish meteorologist
Carl G Rossby in 1939 \parencite{Rossby1939}. Rossby waves are large scale
planetary waves, with typical scales ranging from 
hundreds to thousands of kilometres. The rossby wave phenomenon arise due to the
latitudinal variation in the Coriolis acceleration, which is large at the
high latitudes and zero at the equator. Rossby waves have scales of
hundreds to thousands of kilometres and play a major role in dictating
the weather on daily and large time scales and are crucial for the meridional
transport of heat, moisture and momentum \parencite{midSynDyn}.  

In this study we will examine Rossby waves in a simplified 
atmosphere/ocean. We will assume that our atmosphere/ocean is barotropic, that
is the density depends only on pressure, which via the ideal gas law holds that
the temperature is constant on a isobaric surface, i.e. no stratification. We also assume the flow to be frictionless and
non-divergent, consequently neglecting vertical velocities. Under these
simplifications the quasi geostrophic vorticity equation is reduced to
\cref{eq:QG_vorticity}, where $\zeta$ is the relative vorticity and $f$ is the
planetary vorticity. 
\begin{equation}\label{eq:QG_vorticity}
    \frac{D(\zeta + f)}{Dt} = 0
\end{equation}
A feature of our simplified of the quasi geostrophic vorticity equation is that
change in absolute vorticity $(\zeta + f)$ is zero, in other words the absolute
vorticity is conserved. If a fluid parcel is moving
poleward towards region of larger $f$ then, there must be a compensating change
in $\zeta$ in order for the absolute vorticity to remain unchanged. This
balancing act between planetary and relative vorticity is the essential
mechanism behind
Rossby waves. We will derive the barotropic Rossby
wave equation (BRWE) from \cref{eq:QG_vort} in the subsequent section.

After we have developed our analytical framework, we will discretize the BRWE
and solve the BRWE in both one and two dimension, using both periodic and
bounded boundary condition. We will examine both the forward
difference and central difference scheme and consider their stability.  