\newpage
\section{Results}


All figures shown in this section was produced using the CTCS scheme with
$\Delta x = \frac{1}{40} $ and $\Delta t = $.

\subsection{Periodic}

With periodic boundaries and a sinusoidal inital state \cref{fig:periodic_sine}
the phase speed is constant and all waves move westward. By following one wave
from east to west we can find the phase speed as $\frac{1}{160} = 0.00625$.

\begin{figure}[htp]
  \centering
  \includegraphics[width=\textwidth]{../figures/psi_periodic_centered_short.pdf}
  \caption{Hovmuller diagram for the periodic domain with a sinusoidal initial
  state using a central difference in time. Waves propagate westward.}
  \label{fig:periodic_sine}
\end{figure}



\begin{figure}[htp]
  \centering
  \includegraphics[width=\textwidth]{../figures/hovmuller_sigma.pdf}
  \caption{Hovmuller diagram for the periodic domain with a Gaussian initial
  state.
  From top left to bottom right: $\sigma = \brak{0.08, 0.10, 0.11, 0.12}$.
  }
  \label{fig:periodic_gauss}
\end{figure}



\subsection{Bounded}

With no flow at the boundaries and an initial sinusoidal \cref{fig:bounded_sine}
we can use the same method as for the periodic to find a phase speed of
$\frac{1}{100 - 20} = 0.0125$.


\begin{figure}[htp]
  \centering
  \includegraphics[width=\textwidth]{../figures/psi_bounded_centered_sine.pdf}
  \caption{Hovmuller diagram for the bounded domain with a sinusoidal initial
  state.}
  \label{fig:bounded_sine}
\end{figure}


\begin{figure}[htp]
  \centering
  \includegraphics[width=\textwidth]{../figures/psi_bounded_centered_gauss.pdf}
  \caption{Hovmuller diagram for the bounded domain with a Gaussian initial
  state.}
  \label{fig:bounded_gauss}
\end{figure}


\subsection{2 dimensions}

In two dimensions we only looked at the sinusoidal initial state.
With periodic boundaries \cref{fig:2d_periodic} the waves seem to be travelling
in very much the same way as in the one dimensional case.


\begin{figure}[htp]
  \centering
  \includegraphics[width=\textwidth]{../figures/periodic_2d.pdf}
  \caption{}
  \label{fig:2d_periodic}
\end{figure}



\begin{figure}[htp]
  \centering
  \includegraphics[width=\textwidth]{../figures/bounded_2d.pdf}
  \caption{}
  \label{fig:2d_bounded}
\end{figure}
