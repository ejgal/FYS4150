\section*{Results}

\begin{table}[htp]
  \centering
  \csvautotabular{../processed_data/algorithm_time.csv}
  \caption{Summary of algorithm timings in seconds. Average timings of 10 runs.}
  \label{table:normalized}
\end{table}




To see how the timing of each algorithm scaled with n all the
all timings are divided by n (\cref{table:normalized}). Both TDMA and
TDCMA run times are of the same order, as was
expected from the counting of flops. The run times for LU show an increase of two
orders of magnitude for each magnitude increase in n. This is consistent with
our expectations of the LU algorithm time being proportional to $n^3$.





\begin{figure}[htp]
  \centering
  \includegraphics[width=0.66\textwidth]{../figures/relative_error.png}
  \caption{Maxmimum relative error between analytical and TDMA solution.}
  \label{fig:a}
\end{figure}




\begin{table}[htp]
  \centering
  \csvautotabular{../processed_data/algorithm_time.csv}
  \caption{Summary of algorithm times.}
  \label{table:summary}
\end{table}


To see how the algorithm time for our different methods scales with n we divide
all timings with n (\cref{table:normalized}). Both TDMA and TDCMA run times
are of the same order, as was expected from
the counting of flops. The times for LU show an increase of two orders of magnitude
for each magnitude increase in n. This is consistent with our expectations
of the LU algorithm time being proportional to $n^3$.

\begin{table}[htp]
  \centering
  \csvautotabular{../processed_data/normalized.csv}
  \caption{Algorithm times divided by n.}
  \label{table:normalized}
\end{table}


Comparing the algorithm times of thomas and toeplitz (\cref{table:comparison})
we see they are the same order of magnitude. Theoretically we would
expect the toeplitz algorithm to be $\frac{9 FLOPS}{4 FLOPS} \approx 2.25$ times as fast as
toeplitz, and our results for larger values of n are quite close.

\begin{table}[htp]
  \centering
  \csvautotabular{../processed_data/comparison.csv}
  \caption{Algorithm times compared to TDCMA.}
  \label{table:comparison}
\end{table}


\begin{figure}[htp]
  \centering
  \includegraphics[width=0.66\textwidth]{../figures/TDMA.png}
  \caption{Comparison of analytic solution and numerical approximations.}
  \label{fig:a}
\end{figure}
