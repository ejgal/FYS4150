\section*{Abstract}
This work aims to evaluate Monte Carlo Integration (MCI) for solving physical
problems with many degrees of freedom. The physical problem in question is
the ground state correlation energy between two electrons in a helium atom,
which we solved using four different approaches, "brute force" MCI, MCI with
importance sampling, Gaussian Quadrature (GQ) with Legendre polynomials and GQ
combining Legendre- and Laguerre polynomials. We also parallelized the code.
We did an extensive analysis of the error,
CPU time and scalability of the different approaches. The result of our
analysis showed that we achieved a close to optimal speedup for MCI with
importance sampling. For runtimes larger than 10 seconds MCI with importance
sampling had an error that was between 10 and 383 times smaller than the
GQ Laguerre. Importance sampling MCI had also on average 57 times lower error
that brute force MCI. 
