\section{Results}

\subsection{Monte-Carlo integration}

\subsubsection{Error and accuracy}

\Cref{fig:mc_error} shows the absolute value of the difference between the
calculated value of the integral and the analytical solution with brute force
and importance sampling. N is the number of samples taken. It is worth to notice the sometimes
large differences in error from one run to the next, due to the stochastic nature of
MCI. Still the improvement in accuracy when using importance sampling
is significant especially for larger samples sizes when the standard
deviation is smaller. The mean of the error ratios between brute force and
importance sampling was 57.


\begin{figure}[H]
  \centering
  \includegraphics[width=0.8\textwidth]{../figures/mc_error.png}
  \caption{Absolute value of difference between Monte-Carlo results and analytical
  solution using brute force and importance sampling. Also shows ratio of error
  and the mean of the error ratio.}

  \label{fig:mc_error}
\end{figure}


\Cref{fig:mc_std_time} shows how the standard deviation evolves with increase in
sample size. For large sample sizes we see that the standard deviation decreases
according to what we would expect from the theoretical expression for the standard
deviation \cref{eq: std}, to reduce the standard deviation by a factor 10 you have increase the
sample size by a factor of 100.
\Cref{fig:mc_std_time} also show a reduction of the standard deviation by a
factor of 10 for importance sampling MCI compared to brute force MCI.

\begin{figure}[H]
  \centering
  \includegraphics[width=0.8\textwidth]{../figures/mc_std_time.png}
  \caption{Standard deviation versus run time of MCI experiments.}

  \label{fig:mc_std_time}
\end{figure}

\subsubsection{Parallelization}

\Cref{fig:mc_time_ratio} highlight the benefit of parallelization, showing a
reduction in run time by around a factor 4.
Which is more or less what we would expect
by the specifications of our laptop.

\begin{figure}[H]
  \centering
  \includegraphics[width=0.8\textwidth]{../figures/mc_time_ratio.png}
  \caption{Run time of MCI using importance sampling. Timing of unparallelized
  version divided by run time of the parallelized version.}

  \label{fig:mc_time_ratio}
\end{figure}


\subsection{Gaussian Quadrature}

The error of the GQ methods \cref{fig:gauss_error} shows that GQ Legendre gives
relatively low(high) errors when the number of integration points are odd(even).
For GQ Legendre the error seems to converge around roughly \num{ 3e-3}. The GQ Laguerre shows
a huge decrease in the error with between 1 and 15 integration points, with
the smallest errors at N = 14 and 15. For higher Ns the error
increases and seem to converge to \num{2e-3}, and does not show the oscillatory
nature of the GQ Legendre.

\begin{figure}[H]
  \centering
  \includegraphics[width=0.8\textwidth]{../figures/gauss_error.png}
  \caption{Absolute value of difference between GQ results and analytical
  solution using Legendre and Laguerre polynomials.}

  \label{fig:gauss_error}
\end{figure}


\subsection{Comparison of Gaussian Quadrature and Monte-Carlo Integration}

To decide whether running Gaussian Quadrature or Monte-Carlo Integration would
give the best results for a given run time we made error run time plots for all
of the methods we have used \cref{fig:time_compare}. For MCI low values of run time
corresponds to few samples, and for GQ it corresponds to a small number of integration points.
For the shortest run times the results of MCI were a bit all over the place, but
looking at longer run times we see a clear trend of the error decreasing, especially
for the importance sampling.

We also did a linear interpolation of the error run time results in the range
where we had results for both importance sampling and GQ Laguerre. Taking the rolling
mean of three and three values for the error ratio of GQ Laguerre and importance sampling
gave, for run times larger than 10 seconds, ratios between 20 and 383.

\begin{figure}[H]
  \centering
  \includegraphics[width=0.8\textwidth]{../figures/time_compare.png}
  \caption{Absolute values of the error as a function of time for all methods we
  used to approximate the integral. All of the timings are taken from parallelized
  versions of the code.}

  \label{fig:time_compare}
\end{figure}
